\section{Sets}

It turns out that the mistake was in assuming that we can make a set of all ordinals.
Specifically, we used the ``fact'' (in quotes because it leads to a contradiction) that we can make a set of all objects that satisfy some proposition for any proposition.

In fact, we can more generally see that this leads to contradiction by considering another set, formed in the same way: the set of all sets that do not contain themselves.
Let's call this set $X$ for now.
Is $X \in X$?
Suppose that $X \in X$: but then $X$ is a set that contains itself, so it is \textbf{not} in $X$.
Okay, then $X \not\in X$.
But then, by definition of $X$, we \textbf{do} know that $X \in X$!
Either way, we contradict ourselves.
This is known as Russell's Paradox, named after Bertrand Russell, a mathematician from the early 1900s.

There are several possible resolutions to this paradox: one is to restrict where we can take our objects from, and the other is to restrict which propositions we can use to build new sets.
The first is the more common approach, but some nonstandard set theories allow for the second approach (see: Quine's ``New Foundations'') \reed{Cite} \reed{Should we even mention this?}

In fact, what even is a ``set''?
We can't just build sets however we want without getting a contradiction, as we discovered.
So what \textbf{exactly} can we do without contradiction?

A set of rules, usually called \emph{axioms}, that doesn't allow us to prove a contradiction is called \emph{consistent}.
Let's ask the most basic question: can we be sure that we can't prove a contradiction in any system?

The answer is yes: if we make no assumptions, other than the assumptions of basic logic, we will not be able to prove a contradiction.
This raises the question: what are the rules of the so-called ``basic logic''?
\reed{Should we discuss this? This is basically the ``truth'' section?}

\subsection{Facts about sets}

What sort of rules can we be sure are okay to say about sets?

Let's first talk about some definitional properties of sets: when we talk about sets, what sorts of things might we expect to be true?
You might be thinking of several things, but the first question to ask about objects in general is ``do these objects even exist?''
Not in the physical sense, but as in, is it possible to have such an object without having a contradiction?

For example, there is no number $x$ such that $x = 1$ and $x = 2$, because $1 \neq 2$.
So if we were to start talking about numbers that have the property of being equal to both $1$ and $2$, you would rightly protest because there aren't any numbers.

However, we have no complaints about sets, because we intuitively believe that sets exist.
So let's list that as an axiom, meaning that we can't prove it, but we just assume it's true.
You might protest that we \textbf{can} actually prove that sets exist: after all, isn't $\curlys{1,2,3}$ a set?
But is it?
How do you know it's a set?
Without a definition of ``set'', you can't prove it.
So the existence of sets is a fundamental statement whose truth we assume.
If we assume a set exists, which set is it?
We have no idea, and it's not really important, it just matters that there is some set.
Let's just say that it satisfies the most basic property, which we want every set to satisfy: $x = x$.

\begin{axiom}[Existence]
    There is a set $x$ so that $x = x$.
\end{axiom}

We can also write the previous axiom in symbols as $\exists x (x = x)$, sometimes written as $\exists x \tsuchthat x = x$.

