\section{Why study math?}

Instead of answering the question directly, let's first pose another question: ``What is math?''

Some would say \reed{probably should try to find some actual quotes from real people} that it's about...

And there is really no ``right answer'' to the question ``What is math?''
But I have my answer, and I hope to convince you that it is, at least, an interesting answer.

Math is the study of truth, in the most fundamental way possible.
Math provides the surest path to truth, because it provides \textbf{absolute} truth.
Any theorem is true.
It's true whether the sky is blue or orange or black, whether we're on Earth or on the Moon, whether we live in four dimensions or six or three or two and a half, whether everything is a computer simulation, whether we exist at all.

The sciences may represent and predict reality, but math can do more: it can describe universes that nobody has ever thought of, seen, or phsyically experienced.
We constantly find (or invent, depending on your perspective) new mathematical universes to explore.

What does truth even mean?
How do we know something is true?
How can we understand infinity?
How can we even hope to understand numbers?
All of these questions can be answered by math, and the main method of mathematics is: ``assuming $A$ we can be \textbf{sure}, absolutely sure, that $B$ is true.''

Perhaps this seems like a limiting form, but is there any other way to have truth?
Even the logical rules that we take for granted are nothing but assumptions: there's no proof of modens ponens.
It seems completely impossible that we can have truth without \textbf{some} assumptions, and whether it is or not isn't even the point; this form is a rich world to explore, where anything we can imagine can be tinkered with and there's always another possibility just around the corner.

\reed{The intersection of intuition and precision is the wonder of math, because all mathematical truths are tautologies}

