\section{Ordinals}

The most natural way to represent the size of a finite set is by saying how many elements it has; this number is always a natural number.
Ideally, we'd also have some ``numbers'' that let us discuss the sizes of infinite sets.

But have you ever wondered what numbers are?
As in, \textbf{what is zero?}
\textbf{What is one?}
Or why does $1 + 1 = 2$?
And why does $4 + 2 = 4 + 2$?
And why does that work for all numbers?
Does it work for all numbers?

Well, I won't answer all those questions necessarily, but I will answer some of them.

\subsection{Defining the natural numbers}

Hopefully you aren't tremendously disappointed, but the objective of math is rarely to answer the question ``What \textbf{really} is $X$?''
Instead, we generally ask: ``What are the essential properties of $X$?''
We can also choose to define representations of objects, and if these representations conform to all the properties, then we may just end up saying that they ``really'' \textbf{are} the object, if it becomes convenient.
This, for us, is one of those cases.

Let's start defining some natural numbers.
What are the most basic objects we can talk about, without needing to rely on any other definitions?
We can't say something like ``$3$ means $3$ apples'' or something, because then we need to define apples.
We could just say $0$ is some object with some properties (perhaps that $0 + x = 0$), but we already know of some object that, in many ways, acts just like $0$ does: the empty set.
If I take any set, and I take all the elements in that set and I add the elements of the empty set, then I have the same elements as I did before.
So this property is a lot like the fact that we all know and love, that $0 + x = 0$.
So we'll define $0$ to be $\emptyset$; we write $0 := \emptyset$ to mean that $0 = \emptyset$, and that this is true \emph{by definition}.

This is a fundamental principle of this business of precisely defining objects: there are some properties we want to preserve, and we looks for simpler objects that we already know that have these same properties.
Along the way, we may need to define new notions \reed{should I use a word other than ``notions'' here...} and objects; this is more generally part of the process of mathematical modeling.

We've defined one natural number, the smallest one, which is a good start, but what about the rest of the natural numbers?
Let's consider one other property of the natural numbers: we can get any natural number by just adding $1$ to $0$ a bunch of times.
For example, $4 = 0 + 1 + 1 + 1 + 1$, and $5 = 0 + 1 + 1 + 1 + 1 + 1$.
You can see how this pattern will continue; so I won't write them all out.
So really, we don't actually need to write every single number out, we just need to define what it means to ``add $1$'' to a natural number, and we can get all the rest of the natural numbers.
And to do this, we'll consider one last property of natural numbers: specifically, how do we order natural numbers?
That is, we know that $0 < 1 < 2$, and $0 < 1 < 2 < 3$, and so on.
So the set of natural numbers less than $3$ is $\curlys{0,1,2}$, the set of natural numbers less than $2$ is $\curlys{0,1}$, the set of natural numbers less than $1$ is just $\curlys{0}$, and finally the set of natural numbers less than $0$ is $\emptyset$, because there are no natural numbers less than $0$.
But wait!
We defined $0$ to be $\emptyset$; what if we just defined $1$ to be $\curlys{0}$, and $2$ to be $\curlys{0,1}$, and so on.
That is, each natural number is the set of numbers that are less than it, and ``adding one'' means including the currently number in this collection.

To write the previous sentence in symbols, let's define $n + 1 := n \cup \curlys{n}$.
What does $\cup$ mean?
$\cup$ is used to write the \emph{union} of two sets.
Below is a precise definition of what that means:

\begin{definition}
    For any sets $A$ and $B$, the set $A \cup B$ is the set of all elements that are in either $A$ or $B$, or both.
    That is, $x \in A \cup B$ if and only if $x \in A$ or $x \in B$.
\end{definition}

For example, $\curlys{1,2} \cup \curlys{3,4} = \curlys{1,2,3,4}$, and $\curlys{a,b,c} \cup \curlys{a,d,f} = \curlys{a,b,c,d,f}$.
Note that sets never contain ``repeats'': we can specify a set by just saying what's inside it, without specifying \textbf{how many} of that thing are in the set.

Let's verify that this definition actually works.
We already agreed that $0 = \emptyset$, so then:
\begin{align*}
    0 + 1 &= \emptyset \cup \curlys{\emptyset} = \curlys{\emptyset} = \curlys{0} \\
    1 + 1 &= 1 \cup \curlys{1} = \curlys{\emptyset} \cup \curlys{\curlys{\emptyset}} = \curlys{\emptyset, \curlys{\emptyset}} = \curlys{0, 1} \\
    2 + 1 &= 2 \cup \curlys{2} = \curlys{\emptyset, \curlys{\emptyset}} \cup \curlys{\curlys{\emptyset, \curlys{\emptyset}}} = \curlys{\emptyset, \curlys{\emptyset}, \curlys{\emptyset, \curlys{\emptyset}}} = \curlys{0, 1, 2} \\
    &\vdots
\end{align*}

This definition has the nice property that if $n \in m$, then $n < m$: for example $0 \in 1$, because $0 = \emptyset$ and $1 = \curlys{\emptyset}$.
In fact, because we're trying to define the natural numbers ``from scratch'', this will work for us as a \textbf{definition} of what it means for one natural number to be less than another.

This is the beginning of a definition of the natural numbers, but it is important to note that this isn't the \textbf{only} definition of the natural numbers; there are many, (essentially) equivalent definitions which are appropriate for different situations.

\subsection{Defining arithmetic on natural numbers}

Let's step back for a moment, and take stock of what we've done:

\begin{enumerate}
    \item Defined $0 := \emptyset$
    \item Defined, for any natural number $n$, $n + 1$
    \item Defined what is means for two natural numbers $n$ and $m$ to be less than each other, specifically $n \in m$ means that $n < m$.
\end{enumerate}

What other things might we like to do with natural numbers?
Naturally, we might like to add them, multiply them, and so on.
Let's define addition first, again, being careful to do so only in terms of things we have already defined.
We'll define addition of two natural numbers, $n$ and $m$, as follows:
\begin{align*}
    n + 0 &:= n \\
    n + (m + 1) &:= (n + m) + 1 \\
\end{align*}

Above, we've defined what it means to do $n + 1$, and we just declare that $n + 0 := n$.
Let's do an example of addition:
\[
    1 + 2 = 1 + (1 + 1) = (1 + 1) + 1 = 2 + 1 = 3
\]

Basically, our objective is to ``convert'' the operation of doing $n + m$ into just doing $n + 1 + 1 + \cdots + 1$, because we know how to add $1$ to a natural number.
If you've ever heard ``addition just counting'' \reed{I think there's some more common way that this usually gets phrased}, well, that's true for natural numbers, and that's how we choose to define addition.
And if you've ever heard ``multiplication is just repeated addition'', you'll know how we're going to define multiplication.

\begin{align*}
    n \cdot 0 &:= 0 \\
    n \cdot (m + 1) &:= n + n \cdot m
\end{align*}

\reed{show that we can take max and min by using unions and intersections}

\subsection{Some properties of natural numbers}

Let's step back again, now that we've defined a whole bunch of operations on natural numbers, and look at some properties of natural numbers.
Let's write out a few natural numbers so we can see some patterns:

\begin{itemize}
    \item[] $0 = \emptyset$
    \item[] $5 = \curlys{0,1,2,3,4}$
    \item[] $6 = \curlys{0,1,2,3,4,5}$
    \item[] $10 = \curlys{0,1,2,3,4,5,6,7,8,9}$
\end{itemize}

Specifically, there's a couple important properties here: which we can see what we look at $5$ and $6$.
In fact, maybe it's so obvious that it doesn't even seem to warrant mention, but let's not that not only do we have $5 \in 6$, we also have $5 \subseteq 6$!
Said another way, this is somewhere less surprising: we know that $2 < 5$, and $5 < 6$, so this is just saying that $2 < 6$.
Put yet another way, we know $2 \in 5$, and $5 \in 6$, and $5 \subseteq 6$, so $2 \in 6$.

If you're familiar with the transitive property in general, that if $x < y$ and $y < z$, then $x < z$, you'll recognize this as being something similar.
In fact, let's make preicse what we've been saying, and come up with a general definition of ``transitive'' as it relates to sets:

\begin{definition}
    A set $X$ is called \emph{transitive} if for any $Y \in X$, $Y \subseteq X$.
\end{definition}

Let's now discuss perhaps the most important property of the natural numbers, called \emph{well-ordering}, which says that any subset of the natraul numbers has a smallest element.
To see why this statement is important, or indeed, why this statement really has any meaning at all, let's consider some sets for which it is not true.

\begin{example}
    The set of integers, $\integers$, is not well-ordered, because there is no smallest integer.
\end{example}

In retrospect, this maybe seems obvious, and is probably not what you were thinking of.
Perhaps you were picturing a \emph{bounded} set, such as the integers greater than $-10$ and less than $10$ (or even just greater than $-10$), which certainly does have a least element.
However, if we consider the real numbers, taking again our favorite example---the real numbers between $0$ and $1$ (not including $0$)---we can again see that there even though this set is bounded, both above and below \reed{Should we define these terms, or avoid them, or just not worry about it?}, there is no least element, because we can always halve a number and get a smaller number that's still in the set \reed{is this 100\% obvious? does it merit more explanation}
Now that you're perhaps convinced why well-ordering is a somewhat special property \reed{should we do more on this topic?}, let's give it a proper definition.

\begin{definition}
    A set $A$ is well-ordered if every nonempty subset $B$ of $A$ has a least element.
\end{definition}

Not only are the natural numbers well-ordered, but every natural number is itself well-ordered: any nonempty subset of a natural number $n$ is a subset of the natural numbers.
\reed{elaborate?}

\reed{We should probably talk a little bit more about something stuff here, or give more examples}

\subsection{Ordinals}

Let's expand our definition of natural numbers a little bit, because we want to talk more generally about the sizes of sets, and natural numbers only let us discuss the sizes of finite sets.
That is, we need to start defining infinite numbers.
Despite what you may have heard, it is absolutely possible to talk coherently about infinity, and even infinitely large numbers: in fact, we've already discussed the topic of infinity quite frequently, in discussing the various sizes of infinite and schemes for comparing different infinite sets.

First, let's look at what we did before, defining a natural number to be the set of all natural numbers less than it, and let's look at other things we already know about; specifically, the set of natural numbers, $\nat$, itself.
Firstly, every natural number is a set of natural numbers: that is, for every $n \in \nat$, we also have $n \subseteq \nat$: so $\nat$ is transitive!
Additionally, we already know that $\nat$ is well-ordered. \reed{Should we like, prove this at some point, or?}

So $\nat$ has two of the important properties of natural numbers.
That is, for our purposes, it's somewhat natural number-like.
However, it'd be quite misleading to call it a \textbf{natural number}, so we'll invent a new term for this sort of number: they're called \emph{ordinals}, and they're traditionally denoted by Greek letters, starting with $\alpha$, $\beta$, and $\gamma$, so we'll follow that tradition.
\reed{At some point should we describe why following conventions like this is important? That's probably an important thing, but not sure if it needs to be made explicit. in general, we should tlak about notation and stuff}

\begin{definition}
    A set $\alpha$ is called an \emph{ordinal} if it is transitive and well-ordered by $\in$.
\end{definition}

As we have already seen, all natural numbers are ordinals.
However, $\nat$ is an ordinal, and not a natural number, so not \textbf{all} ordinals are natural numbers.
Additionally, because every natural number is an element of $\nat$, $\nat$ is an ordinal bigger than any other natural number.
That is, $\nat$ is a sort of infinite number: when we want to emphasize that $\nat$ is a infinite ordinal, we'll usually write $\omega_0$ instead of $\nat$, but they're the same thing.

This also brings up to one more incredibly important realization.
For all the natural numbers, we could get to any other natural number just ``adding one'', which we'll call the \emph{successor} from now.
However, this is obviously not the case for $\omega_0$: if we have some natural number $n$, we can't add $1$ any finite number of times to get $\omega_0$.
Put another way, for any natural numbers $n$ and $m$, $n + m < \omega_0$.
So then $\omega_0$ is not the successor of \textbf{any} ordinal!

This leads us to the realization that there are multiple, natural, classes of ordinals: successor ordinals, like the natural numbers, and limit ordinals, like $\omega_0$.
In fact, there's a third kind: $0$.
$0$ isn't the successor of any ordinal, because it's the smallest ordinal, but it's also not really a limit in the same way that $\omega_0$ is, so it get's it's own class.
Let's give some more precise definitions for the various types of ordinals:

\begin{definition}
    An ordinal $\alpha$ is a \emph{successor} ordinal if there is some ordinal $\beta$ so that $\alpha = \beta + 1$.
\end{definition}

\begin{definition}
    An ordinal $\alpha$ is call a \emph{limit} ordinal if for every $\beta < \alpha$, $\beta + < \alpha$.
\end{definition}

So we know that every natural number is an ordinal, and the set of natural numbers is an ordinal...what other ordinals are there?
Well, our previous strategy for building more ordinals was to just take successors.
So why not try that again?
What is $\omega_0 + 1$?

Well, by definition, it's $\omega_0 \cup \curlys{\omega}$, which certainly is \textbf{something}...but is it an ordinal?
The question I'm asking might be phrased more mathematically as: ``is the successor of an ordinal still an ordinal?''
The answer is ``yes'', and the reason will be laid out in the following proof.
Note that we call this result a ``lemma'' rather than a ``thereom'' because it is not a huge, wide reaching fact, but it is a small fact that is often very useful. \reed{Should we clarify?}
\reed{At some point, we should move over to more proofs and stuff, but when should that be?}

\begin{lemma}
    For any ordinal $\alpha$, it's successor, $\alpha + 1$, is also an ordinal.
    \reed{this is a rather formal/abstract thing, maybe we don't need to worry about proving it right now}
\end{lemma}
\begin{proof}
    We need to show both that $\alpha + 1$ is transitive and that it is well-ordered by $\in$, meaning that any nonempty subset of $\alpha$ has a least element.

    Let's first show that it is transitive.
    To do so, we need to show that any element of $\alpha + 1$ is a subset of $\alpha + 1$, so for any $\beta \in \alpha + 1$, $\beta \subseteq \alpha + 1$.

    Suppose that we have some element $\beta \in \alpha + 1$.
    Then because $\alpha + 1 = \alpha \cup \curlys{\alpha}$, either $\beta \in \alpha$ or $\beta \in \curlys{\alpha}$, by definition of the union.
    Let's consider these two cases separately.
    If $\beta \in \alpha$, then we know that $\beta \subseteq \alpha$, because $\alpha$ is an ordinal and therefore $\alpha$ is transitive.
    Also, we know that $\alpha \subseteq \alpha + 1$, because $\alpha + 1 = \alpha \cup \curlys{\alpha}$; and that's all we wanted to show for now.
    \reed{Should we explain/prove that $A \subseteq A \cup B$?}
    Now let's suppose that $\beta \in \curlys{\alpha}$; but that means that $\beta = \alpha$, because $\curlys{\alpha}$ only has one element.
    But as we just discussed, $\alpha \subseteq \alpha + 1$, so $\beta \subseteq \alpha + 1$, just like we wanted.

    Now let's show that $\alpha + 1$ is well-ordered by $\in$.
    That means that we need to show that for any nonempty subset $Y \subseteq \alpha + 1$, there is some $\in$-least element of $Y$.
    \reed{Explain $\in$-least?}

    So let's suppose we have some nonempty subset $Y \subseteq \alpha + 1$, and again we'll consider two cases, either $\alpha \in Y$ or $\alpha \not\in Y$.
    Let's suppose that $\alpha \in Y$.
    Now let's consider two more cases: either $Y = \curlys{\alpha}$, or there is some other element of $Y$, which we'll call $\beta$.
    If $Y = \curlys{\alpha}$, then $\alpha$ is the $\in$-least element of $Y$ because it's the only element of $Y$.
    Otherwise, there must be some other element $\beta \in Y$, that means that $\alpha$ and $\beta$ are different.
    In this case, $\beta < \alpha$, because $\alpha$ is the biggest element in $\alpha + 1$ \reed{Does this need more explanation?}.
    So then the least element of $Y$ must be less than $\alpha$, so the least element of $Y$ must be the same as the least element of $Y$ without $\alpha$.
    However, $Y$ without $\alpha$ is a subset of $\alpha$ \reed{explain?}, so because $\alpha$ is well-ordered by $\in$, there is some $\in$-least element of $Y$.
    Finally, let's consider the case that $\alpha \not\in Y$.
    In this case, we know that $Y \subseteq \alpha$, so again using the fact that $\alpha$ is well-ordered by $\in$, as it is an ordinal, we can use that to get our least element of $Y$.

    This completes, the proof: the successor of every ordinal is also an ordinal!
\end{proof}

This proof means there are \textbf{lots} of ordinals!
In fact, we have ordinals $\omega_0$, $\omega_0 + 1$, $\omega_0 + 2$, $\omega_0 + 3$, and so on.
But how many ordinals are there?
Let's consider the following set: $\omega_0 \cup \curlys{\omega_0, \omega_0 + 1, \omega_0 + 2, \ldots}$, that is, the set of natural numbers unioned with $\omega_0 + n$ for every natural number $n$.
Is this set an ordinal?
It is certainly transitive, because anything in it is either a natural number or $\omega_0 + n$ for some natural number; the natural numbers are all subsets of $\omega_0$, and $\omega_0 + n$ is a subset of $\omega_0 + (n + 1)$.
\reed{Is this proof too quick?}
But is it well-ordered by $\in$?
Indeed, it is: suppose we have some nonempty subset $A$ of this set; because it's nonempty there must be \textbf{some} $\alpha \in A$.
If $\alpha = \emptyset$, then it is the least element, because nothing is less than $0$ (in the natural numbers/ordinals).
However, if it is not, then let's consider the set $\alpha \cap A$: the set of all ordinals that are both in $A$ and $\alpha$.
Note that, because $\alpha$ is an ordinal, there is some least element $\beta \in \alpha \cap A$, and $\beta$ must also be the least element of $A$.
We can see that this last part is true because if there were some other $\gamma \in A$ that was smaller than $\beta$, then $\gamma \in \beta$, and so $\gamma \in \alpha$, so $\beta$ is not the smallest element of $\alpha$ after all.

In fact, this last fact we've proven is quite useful, so let's declare it as a lemma.

\begin{lemma}\label{lem:nonemptyOrdIsOrd}
    If $A$ is a nonempty set of ordinals, then $A$ has a least element.
\end{lemma}
\begin{proof}
    See above, and notice that our proof works for any nonempty set of ordinals, not just the subsets we were considering above.
\end{proof}

This also gives us another useful fact about how to create new ordinals:
\begin{lemma}\label{lem:transSetOfOrdIsOrd}
    If $A$ is a transitive set of ordinals, then $A$ is itself an ordinal.
\end{lemma}
\begin{proof}
    We only need to show that $A$ is well ordered by $\in$, because we already assumed that it is transitive.

    If $A = \emptyset$, then we know that $A$ is an ordinal already.
    If not, then there is some element in $A$, so it is a nonempty set of ordinals.
    If we have some nonempty subset of it, then we know by Lemma \ref{lem:nonemptyOrdIsOrd} that this nonempty subset has a least element.
\end{proof}

Let's talk about one more nice fact about ordinals.
\reed{more motivation}
\begin{lemma}\ref{lem:inOrdIsOrd}
    For an ordinal $\alpha$, any $\beta \in \alpha$ is also an ordinal.
\end{lemma}
\begin{proof}
    Let's take some $\beta \in \alpha$ for some ordinal $\alpha$.
    We want to show that $\beta$ is both transitive and well-ordered by $\in$.

    First, let's show that it is transitive, so if we take some $\gamma \in \beta$, then we want to show that $\gamma \subseteq \beta$.
    Because $\alpha$ is transitive, and $\beta \in \alpha$, we know that $\beta \subseteq \alpha$.
    We also know that $\gamma \in \beta$, so that gives us that $\gamma \in \alpha$ and also that $\gamma \subseteq \alpha$ by using the transitivity of $\alpha$ again.
    We want to know that $\gamma \subseteq \beta$, so let's take the standard approach of taking some $\delta \in \gamma$ and showing that $\delta \in \beta$.
    Note that all three of $\delta$, $\gamma$, and $\beta$ are in $\alpha$, and $\alpha$ is well-ordered by $\in$, so $\delta \in \gamma \in \beta$, which we know, gives us that $\delta \in \beta$.
    \reed{Should probably do more picture proofs or something}

    The last thing is to show that $\beta$ is well-ordered by $\in$, or that every nonempty subset of $\beta$ has a least element.
    However, if we have some nonempty subset $A$ of $\beta$, then $A$ is also a nonempty subset of $\alpha$ because $\alpha$ is a transitive set.
    So then because $\alpha$ is well-ordered by $\in$, there is some least element of $A$, and $\beta$ is also well-ordered by $\in$.
\end{proof}

\reed{Should we construct uncountable ordinals without AXiom of choice (a la Hartogs?) probalby unnecessary, doubt the intended audience is worried about the axiom of choice}
\reed{Here we should eventually define the set of all ordinals, then derive some wacky wacky shit and be like ``wait this isn't right...''}

Let's now consider the set of all ordinals, which we'll call $\bm{Ord}$, and see what kind of statement we can make about this set.
For example, maybe we want to ask how big it is?
Are there as many ordinals as there are real numbers?
What about the powerset of the real numbers, or the powerset of the powerset of the real numbers, and so on?
Certainly it's at least as big as the natural numbers, for several reasons.
For one, it contains the natural numbers; secondly, it is infinite, and we've already seen that every infinite set is at least as large as the natural numbers.

To answer this ``basic'' question, let's notice some facts about this set.
Firstly, every element of this set is an ordinal, so if we have some $\beta \in \bm{Ord}$, then $\beta$ is a ordinal.
This is nothing surprising, because that's literally how we defined the set.
However, notice that, as we just proved in Lemma \ref{lem:inOrdIsOrd}, every element of $\beta$ is also an ordinal.
So that means that every element of $\beta$ is also in $\bm{Ord}$, because that's how we defined $\bm{Ord}$, so then $\beta \subseteq \bm{Ord}$.
But wait: that means that $\bm{Ord}$ is a transitive set of ordinals, and by Lemma \ref{lem:transSetOfOrdIsOrd}, that means that $\bm{Ord}$ \textbf{is} an ordinal!
And that means that $\bm{Ord} \in \bm{Ord}$, so then $\bm{Ord}$ is a set that contains itself!

Is that possible?
On one hand, it seems very strange.
On the other hand, probably so did the fact that there are as many rational numbers as natural numbers.
So maybe our intuition doesn't line up here.
How can we prove that this really is impossible?
And if we can prove this, then that means we just proved a false statement, which definitely can't be done...but where did we go wrong?
At each step, it seems we've been extremely logical, at times, perhaps overly so.
In fact, we can see that this statement is a contradiction: by definition of an ordinal, every ordinal is well-ordered by $\in$.
That means that $\in$ is irreflexive, so $x \not\in x$ for any ordinal $x$.
Except we just showed that $\bm{Ord} \in \bm{Ord}$, and obviously a statement cannot both be true and false. \reed{presumably they will accept this as being ``obvious''}

So we just proved a false statement, using seemingly logical methods.
Does that mean that anything goes?
If we can prove one false statement, can't we prove them all? \reed{Maybe the principle of explosion is not obvious?}

Let's say we're trying to prove some statement $B$, and we know that $A$ is both true and false.
If $B \implies A$, then $\neg A \implies \neg B$, so $\neg B$ is true.
Otherwise, $B$ does not imply $A$, which means that $B$ is true and $A$ is not true.
But we know that $A$ is true, so then $B$ must not be true, so $B$ is false.
So then every statement is false... \reed{idk how convincing this is}

That seems a little hard to accept, so let's explore a more likely solution: we made a mistake in one of the prior proofs?
Which one was it?
\reed{Probably should have people consider this at some point}

