\section{Ordinals}

The most natural way to represent the size of a finite set is by saying how many elements it has; this number is always a natural number.
But have you ever wondered what numbers are?
As in, \textbf{what is zero?}
\textbf{What is one?}
Or why does $1 + 1 = 2$?
And why does $4 + 2 = 4 + 2$?
And why does that work for all numbers?
Does it work for all numbers?

Well, I won't answer all those questions necessarily, but I will answer some of them.

\subsection{Defining the natural numbers}

Hopefully you aren't tremendously disappointed, but the objective of math is rarely to answer the question ``What \textbf{really} is $X$?''
Instead, we generally ask: ``What are the essential properties of $X$?''
We can also choose to define representations of objects, and if these representations conform to all the properties, then we may just end up saying that they ``really'' \textbf{are} the object, if it becomes convenient.
This, for us, is one of those cases.

Let's start defining some natural numbers.
What are the most basic objects we can talk about, without needing to rely on any other definitions?
We can't say something like ``$3$ means $3$ apples'' or something, because then we need to define apples.
We could just say $0$ is some object with some properties (perhaps that $0 + x = 0$), but we already know of some object that, in many ways, acts just like $0$ does: the empty set.
If I take any set, and I take all the elements in that set and I add the elements of the empty set, then I have the same elements as I did before.
So this property is a lot like the fact that we all know and love, that $0 + x = 0$.
So we'll define $0$ to be $\emptyset$; we write $0 := \emptyset$ to mean that $0 = \emptyset$, and that this is true \emph{by definition}.

This is a fundamental principle of this business of precisely defining objects: there are some properties we want to preserve, and we looks for simpler objects that we already know that have these same properties.
Along the way, we may need to define new notions \reed{should I use a word other than ``notions'' here...} and objects; this is more generally part of the process of mathematical modeling.

We've defined one natural number, the smallest one, which is a good start, but what about the rest of the natural numbers?
Let's consider one other property of the natural numbers: we can get any natural number by just adding $1$ to $0$ a bunch of times.
For example, $4 = 0 + 1 + 1 + 1 + 1$, and $5 = 0 + 1 + 1 + 1 + 1 + 1$.
You can see how this pattern will continue; so I won't write them all out.
So really, we don't actually need to write every single number out, we just need to define what it means to ``add $1$'' to a natural number, and we can get all the rest of the natural numbers.
And to do this, we'll consider one last property of natural numbers: specifically, how do we order natural numbers?
That is, we know that $0 < 1 < 2$, and $0 < 1 < 2 < 3$, and so on.
So the set of natural numbers less than $3$ is $\curlys{0,1,2}$, the set of natural numbers less than $2$ is $\curlys{0,1}$, the set of natural numbers less than $1$ is just $\curlys{0}$, and finally the set of natural numbers less than $0$ is $\emptyset$, because there are no natural numbers less than $0$.
But wait!
We defined $0$ to be $\emptyset$; what if we just defined $1$ to be $\curlys{0}$, and $2$ to be $\curlys{0,1}$, and so on.
That is, each natural number is the set of numbers that are less than it, and ``adding one'' means including the currently number in this collection.

To write the previous sentence in symbols, let's define $n + 1 := n \cup \curlys{n}$.
Let's verify that this definition actually works.
We already agreed that $0 = \emptyset$, so then:
\begin{align*}
    0 + 1 &= \emptyset \cup \curlys{\emptyset} = \curlys{\emptyset} = \curlys{0} \\
    1 + 1 &= 1 \cup \curlys{1} = \curlys{\emptyset} \cup \curlys{\curlys{\emptyset}} = \curlys{\emptyset, \curlys{\emptyset}} = \curlys{0, 1} \\
    2 + 1 &= 2 \cup \curlys{2} = \curlys{\emptyset, \curlys{\emptyset}} \cup \curlys{\curlys{\emptyset, \curlys{\emptyset}}} = \curlys{\emptyset, \curlys{\emptyset}, \curlys{\emptyset, \curlys{\emptyset}}} = \curlys{0, 1, 2} \\
    &\vdots
\end{align*}

This definition has the nice property that if $n \in m$, then $n < m$: for example $0 \in 1$, because $0 = \emptyset$ and $1 = \curlys{\emptyset}$.
In fact, because we're trying to define the natural numbers ``from scratch'', this will work for us as a \textbf{definition} of what it means for one natural number to be less than another.

This is the beginning of a definition of the natural numbers, but it is important to note that this isn't the \textbf{only} definition of the natural numbers; there are many, (essentially) equivalent definitions which are appropriate for different situations.

\subsection{Defining arithmetic on natural numbers}

Let's step back for a moment, and take stock of what we've done:

\begin{enumerate}
    \item Defined $0 := \emptyset$
    \item Defined, for any natural number $n$, $n + 1$
    \item Defined what is means for two natural numbers $n$ and $m$ to be less than each other, specifically $n \in m$ means that $n < m$.
\end{enumerate}

What other things might we like to do with natural numbers?
Naturally, we might like to add them, multiply them, and so on.
Let's define addition first, again, being careful to do so only in terms of things we have already defined.
We'll define addition of two natural numbers, $n$ and $m$, as follows:
\begin{align*}
    n + 0 &:= n \\
    n + (m + 1) &:= (n + m) + 1 \\
\end{align*}

Above, we've defined what it means to do $n + 1$, and we just declare that $n + 0 := n$.
Let's do an example of addition:
\[
    1 + 2 = 1 + (1 + 1) = (1 + 1) + 1 = 2 + 1 = 3
\]

Basically, our objective is to ``convert'' the operation of doing $n + m$ into just doing $n + 1 + 1 + \cdots + 1$, because we know how to add $1$ to a natural number.

\reed{show that we can take max and min by using unions and intersections}

\subsection{Some properties of natural numbers}

\reed{Here we should eventually define the set of all ordinals, then derive some wacky wacky shit and be like ``wait this isn't right...''}

