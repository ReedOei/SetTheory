\section{Games}

\reed{intro text here}

\subsection{What is a game?}

When we talk about games, what do we mean?

Well, there's some number of players, who each make moves for some amount of time, until somebody wins.
Many ``real'' games also allow a tie, but ties are fundamentally uninteresting: just think about all the games that have ``tiebreakers'' or some similar mechanic.
Following these games' example, we will consider games in which there is always exactly one winner.
\reed{Define games}

\reed{Give some examples of games}

\begin{proposition}
    Any game player with more than two players can be simulated by a game with only two players.
\end{proposition}
\begin{proof}
    \reed{For each move $a_i$ by a player $p_j$, just create the new move $(a_i,p_j)$. Because moves and games are both natural numbers, still countable, so we can encode}
\end{proof}

\begin{theorem}
    Finite games are determined.
\end{theorem}
\begin{proof}
    Let's introduce some new terminology to simplify this proof.
    This is a common strategy: solving problems often requires new ideas, and as you will see, the right terminology, coupled with the right intuition, often makes the solution to a problem much clearer.

    We will call a position (note that this refers to a position, not the whole game) ``determined'' \reed{can we do better so we don't have to resuse terminology?} if, from that point on, one of the players has a winning strategy.
\end{proof}

