\section{The Kinds of Infinity}

In the previous section, we established that there are multiple kinds, or sizes, of infinity.
Or, at least, there are two: the things that are the same size as the positive integers, and the things that are as many as groups of positive integers.

\subsection{Some New Language}

Our language, while quite versatile, is not quite ideal for the topic at hand.
That previous sentence is probably very difficult to read, and is even more difficult to talk about with other people.
In fact, many of the previous sentences are, despite the author's best effort, difficult to parse.

To fix this problem, we're going to introduce some new words to make talking about groups and infinities a bit easier.

First, we're going to introduce so new terms for things we've been saying; in particular, we want to replace some common English terms with their technical versions, because using plain English terms can be misleading if we don't mean them in a colloquial sense.
Instead of ``group'', we'll use \emph{set} or \emph{collection}.
The things inside a set or collection are called \emph{elements}.
Note that order in a set doesn't matter, the set $\curlys{1,2}$ is the same as the set $\curlys{2,1}$; we only care what's in the set.
Instead of ``amount'', we will use \emph{cardinality}.
Then, instead of saying ``we have the same amount of students in the original class as there are chairs'', we'll say ``the collection of students in the new class has same cardinality as the set of chairs.''
This isn't shorter, but it's more precise: any mathematically trained person will know what we mean; if we use ``group'' and ``amount'', it's too vague to be sure what we're really talking about without going into more detail.
Instead of saying ``the set $A$ has the same cardinality as natural numbers'', we'll say ``$A$ is \emph{countable}'', or ``there are \emph{countably many} elements of $A$''.
If some set is too big to have a good, complete assignment with a countable set, we'll say it is \emph{uncountable}.
Finally, instead of ``groups of things in a set $X$'' we'll say ``\emph{subsets} of $X$''; so then we can say ``the set of subsets of positive integers is uncountable.''
\reed{Is this too many new definitions to just throw at them?}
\reed{Should definitions be more rigorous?}

For some examples, $\curlys{1,2,3}$ is a group of things in the bigger group $\curlys{1,2,3,4,5}$, so $\curlys{1,2,3}$ is a \textbf{subset} of $\curlys{1,2,3,4,5}$.
Similarly, $\curlys{1}$ is a subset of $\curlys{1,2,3,4,5}$.
Finally, note that we consider the ``empty group'' or \emph{empty set}, $\emptyset := \curlys{}$, which has no elements to be a subset of \textbf{every} set.
\reed{How to build familiarity with the empty set and sets in general, but maintaining the flow?}
\reed{Is this even important right now?}

\subsection{How many sizes of infinity?}

As briefly discussed earlier, we've so far found two different kinds of infinity: countable and uncountable.
Is that all there is: specfically, if a set is uncountable, it is the same size as every other uncountable set?
\reed{I have literally no intuition for guessing what people will think about this...}

The answer is no, and we'll need one more term to talk about it: the \emph{powerset} of a set is the set of all subsets of that set.
So the powerset of $\curlys{1,2,3}$ is $\curlys{\emptyset, \curlys{1}, \curlys{2}, \curlys{3}, \curlys{1,2}, \curlys{1,3}, \curlys{2,3}, \curlys{1,2,3}}$.
By convention, for a set $A$, we consider both $\emptyset$ and $A$ to be subsets of $A$.
\reed{Should probably explain this convention...}
We write ``the powerset of a set $A$'' as $\powerset{A}$.
Then we can rephrase the earlier sentence about the powerset of $\curlys{1,2,3}$ as:

\[
    \powerset{\curlys{1,2,3}} = \curlys{\emptyset, \curlys{1}, \curlys{2}, \curlys{3}, \curlys{1,2}, \curlys{1,3}, \curlys{2,3}, \curlys{1,2,3}}
\]

Then we have the following theorem:

\begin{theorem}
    For any set $A$, the cardinality of the powerset of $A$, $\powerset{A}$, is greater than the cardinality of $A$; that is:

    \[
        \abs{\powerset{A}} > \abs{A}
    \]
\end{theorem}

\begin{proof}
    The proof here is almost the same as the previous section, where we showed that adding a new student for each group of students in the original class, or each subset of the students in the original class, gives a set of students that is larger than the original class.

    For the sake of a contradiction \reed{we should explain proof by contradiction at some point}, let's suppose that the two sets have the same cardinality, meaning there is a good, complete assignment of the elements in $A$ to the elements in $\powerset{A}$, which are subsets of $A$.
    That is, for every element $x$ of $A$, there is some subset of $A$, $A_x$; and every subset of $A$ is $A_x$ for some element $x$ of $A$.

    Consider the subset of $A$ which contains $x$ if and only if $x$ is \textbf{not} in $A_x$, and call this subset $B$.
    By our assumption, we must have $B = A_y$ for some element $y$ of $A$.
    But by definition, if $y$ is an element of $A_y$, then $y$ is not an element of $B$; if $y$ is not an element of $A_y$, then $y$ is an element of $B$.
    Therefore, the two sets \textbf{cannot} be the same, so there is some subset, namely $B$, which is missing from our list.
    So there cannot be any such good, complete assignment.
\end{proof}

But this brings up a new, very interesting question.

\begin{quote}
    If we have some set $A$, is there any set that has a cardinality greater than $A$, but less than $\powerset{A}$?
\end{quote}

At first, it may seem like the answer to this question is quite obvious.
If I have a set with $4$ things in it, like $\curlys{1,2,3,4}$, then the set $\curlys{1,2,3,4,5}$ is a bigger set, but it doesn't have as many things as the powerset of the original set.
And, indeed, whenever a set is finite, we can easily find such a set.
The question only really becomes interesting when we talk about infinite sets.

As we've seen, even if we add a single element to an infinite set, the infinite set doesn't become any ``bigger''; that is, the cardinality of the set remains the same.
In fact, even ``doubling'' and ``squaring'' the size of the set didn't make any difference in the cardinality.
The only operation we've seen so far that makes a difference is taking the powerset.
Anything less, and it seems that we can't increase the size of the set by enough to matter.

On the other hand, it seems like there ought to be some set with a cardinality between the powerset and the set itself.
After all, there's a lot of ``space'' in between these sets. \reed{clarify this point}

But it's very difficult to talk about cardinality for completely abstract sets that contain only other abstract objects as elements.
Instead, we'll talk about something much more concrete: numbers.
For the moment, we don't really care \textbf{what} our set contains, only \textbf{how many things} it contains.
So for convenience, we might as well just talk about some well-understood set of things, and numbers fit the bill.

\subsection{Different Kinds of Numbers}

We've discussed and used a variety of different kinds of numbers throughout this book.
First, we talked about natural numbers: $0$, $1$, $2$, $3$, $4$, $5$, etc.

But there are, of course, other kinds of numbers: what about $0$, $-10$, $\frac{1}{2}$, or the famous $\pi$?
Each of these numbers is a representative of a different kind of numbers.
Each kind of numbers forms a set of numbers.

We call the numbers $0$, $1$, $2$, $3$, $4$, $5$, and so on the \emph{natural numbers}, because they are ``natural'' in the sense that they are obviously useful for counting.
We denote the set of all natural numbers by $\nat$, which is just a fancy way to write an $N$.

The numbers like $-10$, $-5$, $2$, or $129$ are called \emph{integers}, coming from the Latin \emph{integer}, meaning ``whole''.
We can clearly see that every natural is itself an integer, but we also include the natural numbers.
\reed{Is this actually that obvious...I think so?}
We write the set of integers as $\integers$, because the German word for ``numbers'' is \emph{Zahlen}.

Next, we have numbers like $\frac{5}{2}$, $-\frac{11}{2}$, $\frac{129}{1819}$, and so on.
These numbers are called \emph{rational numbers}, because they are ``ratios'' of integers (where the bottom number is not $0$).
We write the set rational numbers as $\rational$ for ``quotient''.

And finally, we get to the biggest group of numbers we'll consider for now: the ``real numbers'', written $\real$.
You can think of these as all the ``decimal numbers'': so all the rational numbers, but also numbers like $\pi$ or $\sqrt{2}$. \reed{We should have some introduction to this}
In some sense, the name ``real'' is a bit unfortuante: the real numbers are no more ``real'' than other numbers.
But the name has stuck, and it's hard to imagine changing it to anything else.

\subsection{How many of each kind of number are there?}

The motivation for even discussing numbers in the first place is talking about the size of sets; let's do that.

First, let's establish a baseline: countable sets (those that have the same cardinality as the natural numbers).
We'll compare everything to countable sets, because in some sense, there's no smaller kind of infinite set.
Why is this?

Let's first try out defining what an infinite set is \reed{should have them think about it first}: it's basically something that has lots and lots of things in it.
Or, another way to put it, is that it's not finite.
But then what does being finite mean?

\begin{definition}
    A set $A$ is \emph{finite} if it has the same cardinality as the set $\curlys{0, 1, 2, \ldots, n}$ for some natural number $n$; equivalently, if there is a good, complete assignment between $A$ and $\curlys{0, 1, 2, \ldots, n}$ for some natural number $n$.
\end{definition}

This leads to an easy definition of infinite sets.

\begin{definition}
    A set $A$ is \emph{infinite} if it is not finite.
\end{definition}

Now we can describe why there's no ``smaller kind of infinity'' than countable sets.

\begin{theorem}
    If $A$ is infinite, then it's cardinality is at least as large as the cardinality of the natural numbers, $\nat$.
\end{theorem}
\begin{proof}
    To see that this is the case, we need to build some good, complete assignment between some subset of $A$ and the natural numbers $\nat$, or vice versa; we need to pair each natural number with some element of $A$.

    Let's start with $0$, because it's the smallest natural number.
    Then pick any element from $A$, let's call it $a_0$.
    We'll assign $0$ to $a_0$.

    Now let's move on to the next natural number, $1$.
    Pick some other element from $A$, which is not $a_0$, and call it $a_1$.
    Note that there has to be some element, because $A$ is infinite: if it only contained $a_0$, it would obviously be finite!
    So then we'll assign $1$ to $a_1$.

    Then just repeat this process for each $n$, so we assign $n$ to some element of $A$ that is not any of the previously chosen $a_0$, $a_1$, $a_2$, etc. \reed{Sneaky recursively defined function here}
    At every step there will be some $a_n$ we haven't used, or we'll have built a good, complete assignment between $\curlys{0, 1, 2, \ldots, n - 1}$, but that means that $A$ is finite!
    Note that this assignment is, in fact, good: we only use each element of $a$ once, so we'll never have two different natural numbers assigned to the same element of $A$.

    So then we built a good, complete assignment between $\nat$ and some subset of $A$, namely, the subset consisting of all choices we made during the process described above: that might seem a little circular, but it's not. \reed{explain why not?}
    So then $A$ must be at least as big as $\nat$.
\end{proof}

So that means that every infinite set is at least countable.

Let's talk about the integers, $\integers$, now.
How many of them are there?
At first glance, it seems obviously more than the natural numbers; but recall the argument from before, when we ``doubled'' the size of our class, but ended up with the same cardinality of students?
We can apply that same trick here: we are doubling the natural numbers: for every $n$, we have $-n$ (except $0$).

What about the rationals, $\rational$?
Clearly that must be bigger: but again, we already showed it's not.
Remember that we showed that giving each student in the infinite, countable class results in the same cardinality for the set of students.
And what are rational numbers, other than pairs of integers (one for the numerator, and one for the denominator); technically, there's even fewer rationals than this, because $\frac{1}{2} = \frac{2}{4}$, so there's fewer rationals than pairs of integers! \reed{Should we clarify this point?}
But the set is definitely infinite, because we have one rational for each natural number, specifically, for $n$, we have $\frac{n}{1}$; alternatively, $\nat$ is a subset of $\rational$.
So the rationals are also countable!

What about real numbers?
Surely the real numbers must be bigger than the natural numbers?
Let's just talk about the real numbers that are between $0$ and $1$: there's still a lot of them there.
But how many?
Well, let's talk about powersets again.
If we could show there are as many real numbers between $0$ and $1$ as there are subsets of natural numbers, that would show there are more real numbers between $0$ and $1$ than all of the natural numbers: we showed this earlier.
So we want to pair up real numbers between $0$ and $1$ and natural numbers.
We can do this by writing real numbers as decimals, specifically, decimals in binary. \reed{can we avoid using binary? should we avoid using binary (maybe just teach them binary at some point)?}
Each real number between $0$ and $1$ looks like $0.0101001010\ldots$ for some string of $0$'s and $1$'s.

What is the connection between subsets of natural numbers and binary decimal expansions of real numbers?
We can think of a subset of natural numbers as a collection of numbers, but we can also think of it as a sequence of $0$'s and $1$'s: a binary sequence.
The $n$-th digit of this sequence is $1$ is $n$ is in the subset, and $0$ if it is not.
We can represent every subset this way. \reed{Should we prove this?}
So now we have a binary sequence for each subset of natural numbers.
From here, it's quite straightforward to assign each such sequence to a real number between $0$ and $1$.
Just write $0.$ followed by the corresponding binary sequence, so if your subset only contains $1$ and $2$, then the corresponding real number is $0.0110000\ldots$, where the $0$'s continue forever.

\reed{Should we even talk about this? It's kind of a technicality}
There is one issue: some subsets of natural numbers will map to the same real number.
For example $0.0\overline{1} = 0.1\overline{0}$; that is $0.011111111\ldots$, where the $1$'s continue forever, is equal to $0.100000$, where the $0$'s continue forever. \reed{Definitely will need to prove this at some point...}
However, note that this only occurs when the decimal representation of the real number is finite and followed by infinitely many zeroes.
This happens only for a countable number of real numbers \reed{Should probably prove this too at some point}, so if we remove all of these subsets of natural numbers, we get a good, complete assignment of subsets of natural numbers to real numbers.
Note that after removing these subsets, the resulting set is still not countable; that is, it is strictly larger than the natural numbers, because if it were not, then we could get all the subsets of natural numbers by unioning two countable sets, which is a contradiction. \reed{This needs more detail/explanation probably}

So there are as many real numbers between $0$ and $1$ as there are natural numbers!
So there are, in fact, \textbf{more} real numbers than natural numbers.
This probably conforms with your intuition, and it isn't too surprising, though perhaps seeing that there are as many rational numbers as natural numbers caused you to doubt this.
Speaking of which, this proof actually does something else very interesting: because there are more real numbers than natural numbers, but the same number of rational numbers as natural numbers, ther emust be \textbf{far} more real numbers than there are rational numbers!
That means that \textbf{almost every} real number is \textbf{not} rational!
\reed{This probably merits more discussion}

But we still have one lingering question: how many real numbers are there?
In fact, there are the same number of real numbers, total, as there are real numbers between $0$ and $1$.
We can see this in figure \ref{fig:bijection-01-real}.

\begin{figure}\label{fig:bijection-01-real}
    \centering
    \begin{tikzpicture}
        \node[] (left) at (-6,2) {};
        \node[] (right) at (10,2) {};

        \draw[<->, thick] (left) -- (right);
        \path[name path=line] (left) -- (right);

        \node[circle, fill=black, inner sep=0.1em, label=above:$0$] (zero) at (2,2) {};
        \node[circle, fill=black, inner sep=0.1em, label=above:$1$] (p1) at (4,2) {};
        \node[circle, fill=black, inner sep=0.1em, label=above:$-1$] (m1) at (0,2) {};
        \node[circle, fill=black, inner sep=0.1em, label=above:$2$] (p2) at (6,2) {};
        \node[circle, fill=black, inner sep=0.1em, label=above:$-2$] (m2) at (-2,2) {};
        \node[circle, fill=black, inner sep=0.1em, label=above:$3$] (p3) at (8,2) {};
        \node[circle, fill=black, inner sep=0.1em, label=above:$-3$] (m3) at (-4,2) {};

        \node[circle, fill=black, inner sep=0.1em, label=above:$\sqrt{2}$] (rt2) at (4.91421356237309,2) {};

        \node[circle, fill=black, inner sep=0.1em, label=right:$0$] (x0) at (0,0) {};
        \node[circle, fill=black, inner sep=0.1em, label=right:$1$] (x1) at (4,0) {};
        \node[circle, fill=black, inner sep=0.1em] (center) at (2,0) {};

        \draw[thick, name path=arcpath] (x0) arc [radius=2, start angle=180, end angle=0];

        \draw[thick] (center) -- (m3);
        \path[name path=m3line] (center) -- (m3);
        \path[name intersections={of=arcpath and m3line, by=m3int}];
        \node[circle, fill=black, inner sep=0.1em, label=left:$0.1024\ldots$] (rt2i) at (m3int) {};

        \draw[thick] (center) -- (rt2);
        \path[name path=rt2line] (center) -- (rt2);
        \path[name intersections={of=arcpath and rt2line, by=rt2int}];
        \node[circle, fill=black, inner sep=0.1em, label=right:$0.8040\ldots$] (rt2i) at (rt2int) {};
    \end{tikzpicture}

    \caption{A pairing of the real numbers between $0$ and $1$ and all the real numbers}
\end{figure}

So if there are as many real numbers between $0$ and $1$ as there are real numbers in general, I think you will not have trouble believing me when I say that this is also true for any (distinct) real numbers: if $x$ and $y$ are real numbers, with $x \neq y$, then there are as many real numbers between $x$ and $y$ as there are real numbers in general.
But, as we know, there are \textbf{not} as many rational numbers as there are real numbers.
So far, every infinite set we've only talked about is either countable or has the same cardinality as the real numbers.
Is there anything between these two?

\subsection{The Continuum Hypothesis}

And that, my friends, is the Continuum Hypothesis, or CH.
As you might be able to guess from the title case name, it's a bit of a Big Deal.
Specifically, the Continuum Hypothesis is the following statement:

\begin{quote}
    There are no sets with cardinality strictly greater than the natural numbers and strictly less than the real numbers.
\end{quote}

Depending on who you are, this statement may feel intuitively true, or intuitively false.

On the one hand, we haven't seen any sets that bigger than countable: the rational feel bigger, but they're not.
Even if we pair lots and lots and lots of integers together, and consider all such pairs, we see that it's still countable!
From the other side, if we take anything other than special subsets of the reals like the natural numbers or rational numbers, then we get a set that's as big as all the real numbers.
We saw this when we considered all the real numbers between $0$ and $1$, but of course this will also happen for \textbf{any} interval \reed{Define interval or just don't use the word?}.
So how would anything fit between these potentially infinitely close together reals and the countable sets, which are so much smaller?

But, on the other side, the gap in size between countable numbers and the reals is so \textbf{huge} that it feels like we can easily fit in more cardinalities.
Otherwise, at some point, we might take a single step and cross the gap instantly from countable to \emph{continuum}, the cardinality of the reals.

The other question you might ask, what's bigger than the real numbers, is already answered: the powerset of the reals is bigger.
But this leads to a more general form \reed{should we say what this means...it's not standard English really} of the previous statement:

\begin{quote}
    If $A$ is infinite, then there are no sets with cardinality strictly greater than $A$ but strictly less than $\powerset{A}$.
\end{quote}

This statement is known as the Generalized Continuum Hypothesis, or GCH.
It is perhaps not too surprising that if GCH is true, then CH is also true, because CH is basically just GCH with $A = \nat$

But the most surprising thing of all is that both CH and GCH are \emph{independent} of the standard mathematical axioms!
What does that mean?
It basically means that we cannot prove that CH (and, by extension, GCH) is true; but we also cannot prove that it is false!
Specifically, the standard axiomatic system for mathematics, ZFC, which essentially serves as a foundation for all of mathematics, CH is neither true nor false.
That is, it is consistent with the ZFC axioms that CH is true, but it is also consistent with the ZFC axioms that CH is false (assuming ZFC is itself consistent).

We should take a step back here to realize what a momentous thing this is.
Most people think of mathematics as being a subject in which everything has a definitive answer; even if we don't know the answer now, there must be \textbf{some} answer.
But this is not true.
For some questions, the ``answer'', if you would call it that (I would), is that the question cannot be answered.
We must also underscore that it is not that the Continuum Hypothesis is really hard to prove true, or really hard to prove false, and just haven't figured it out yet: we've (we in the sense of the human race) have \textbf{proven} that it cannot be proven.

