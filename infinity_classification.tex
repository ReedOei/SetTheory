\section{The Kinds of Infinity}

In the previous section, we established that there are multiple kinds, or sizes, of infinity.
Or, at least, there are two: the things that are the same size as the positive integers, and the things that are as many as groups of positive integers.

\subsection{Some New Language}

Our language, while quite versatile, is not quite ideal for the topic at hand.
That previous sentence is probably very difficult to read, and is even more difficult to talk about with other people.
In fact, many of the previous sentences are, despite the author's best effort, difficult to parse.

To fix this problem, we're going to introduce some new words to make talking about groups and infinities a bit easier.

First, we're going to introduce so new terms for things we've been saying; in particular, we want to replace some common English terms with their technical versions, because using plain English terms can be misleading if we don't mean them in a colloquial sense.
Instead of ``group'', we'll use \emph{set} or \emph{collection}.
The things inside a set or collection are called \emph{elements}.
Note that order in a set doesn't matter, the set $\curlys{1,2}$ is the same as the set $\curlys{2,1}$; we only care what's in the set.
Instead of ``amount'', we will use \emph{cardinality}.
Then, instead of saying ``we have the same amount of students in the original class as there are chairs'', we'll say ``the collection of students in the new class has same cardinality as the set of chairs.''
This isn't shorter, but it's more precise: any mathematically trained person will know what we mean; if we use ``group'' and ``amount'', it's too vague to be sure what we're really talking about without going into more detail.
Instead of saying ``the set $A$ has the same cardinality as positive integers'', we'll say ``$A$ is \emph{countable}'', or ``there are \emph{countably many} elements of $A$''.
If some set is too big to have a good, complete assignment with a countable set, we'll say it is \emph{uncountable}.
Finally, instead of ``groups of things in a set $X$'' we'll say ``\emph{subsets} of $X$''; so then we can say ``the set of subsets of positive integers is uncountable.''
\reed{Is this too many new definitions to just throw at them?}
\reed{Should definitions be more rigorous?}

For some examples, $\curlys{1,2,3}$ is a group of things in the bigger group $\curlys{1,2,3,4,5}$, so $\curlys{1,2,3}$ is a \textbf{subset} of $\curlys{1,2,3,4,5}$.
Similarly, $\curlys{1}$ is a subset of $\curlys{1,2,3,4,5}$.
Finally, note that we consider the ``empty group'' or \emph{empty set}, $\emptyset := \curlys{}$, which has no elements to be a subset of \textbf{every} set.
\reed{How to build familiarity with the empty set and sets in general, but maintaining the flow?}
\reed{Is this even important right now?}

\subsection{How many sizes of infinity?}

As briefly discussed earlier, we've so far found two different kinds of infinity: countable and uncountable.
Is that all there is: specfically, if a set is uncountable, it is the same size as every other uncountable set?
\reed{I have literally no intuition for guessing what people will think about this...}

The answer is no, and we'll need one more term to talk about it: the \emph{powerset} of a set is the set of all subsets of that set.
So the powerset of $\curlys{1,2,3}$ is $\curlys{\emptyset, \curlys{1}, \curlys{2}, \curlys{3}, \curlys{1,2}, \curlys{1,3}, \curlys{2,3}, \curlys{1,2,3}}$.
By convention, for a set $A$, we consider both $\emptyset$ and $A$ to be subsets of $A$.
\reed{Should probably explain this convention...}
We write ``the powerset of a set $A$'' as $\powerset{A}$.
Then we can rephrase the earlier sentence about the powerset of $\curlys{1,2,3}$ as:

\[
    \powerset{\curlys{1,2,3}} = \curlys{\emptyset, \curlys{1}, \curlys{2}, \curlys{3}, \curlys{1,2}, \curlys{1,3}, \curlys{2,3}, \curlys{1,2,3}}
\]

Then we have the following theorem:

\begin{theorem}
    For any set $A$, the cardinality of the powerset of $A$, $\powerset{A}$, is greater than the cardinality of $A$; that is:

    \[
        \abs{\powerset{A}} > \abs{A}
    \]
\end{theorem}

\begin{proof}

\end{proof}

\subsection{Different Kinds of Numbers}

We've discussed and used a variety of different kinds of numbers throughout this book.
First, we talked about \reed{positive integers}: $1$, $2$, $3$, $4$, $5$, etc.

But there are, of course, other kinds of numbers: what about $0$, $-10$, $\frac{1}{2}$, or the famous $\pi$?
Each of these numbers is a representative of a different kind of numbers.
Each kind of numbers forms a set of numbers.

We call the numbers $0$, $1$, $2$, $3$, $4$, $5$, and so on the \emph{natural numbers}, because they are ``natural'' in the sense that they are obviously useful for counting.
We denote the set of all natural numbers by $\nat$, which is just a fancy way to write an $N$.

The numbers like $-10$, $-5$, $2$, or $129$ are called \emph{integers}, coming from the Latin \emph{integer}, meaning ``whole''.
We can clearly see that every natural is itself an integer, but we also include the natural numbers.
\reed{Is this actually that obvious...I think so?}
We write the set of integers as $\integers$, because the German word for ``numbers'' is \emph{Zahlen}.

Next, we have numbers like $\frac{5}{2}$, $-\frac{11}{2}$, $\frac{129}{1819}$, and so on.
These numbers are called \emph{rational numbers}, because they are ``ratios'' of integers (where the bottom number is not $0$).
We write the set rational numbers as $\rational$ for ``quotient''.

\subsection{How many of each kind of number are there?}

