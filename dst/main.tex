\documentclass[10pt]{article}

\usepackage{amsmath}
\usepackage{sagetex}
\usepackage{hyperref}
\usepackage{tikz-cd}
\usepackage{amssymb}
\usepackage{amsthm}
\usepackage{bm}
\usepackage{listings}
\usepackage{bbm}
\usepackage{multicol}
\usepackage{mathtools}
\usepackage{mathpartir}
\usepackage{float}
\usepackage{enumitem}
\usepackage[margin=1.25in]{geometry}
\usepackage[T1]{fontenc}
\usepackage{kpfonts}

\usepackage{xcolor}
\usepackage{forloop}
\usepackage{ifthen}
\usepackage{calc}
\usepackage{xspace}
\usepackage{mathrsfs}

\newcounter{i}
\newcounter{j}
\newcounter{n1}

\newcommand{\concat}{%
  \mathbin{\raisebox{0.5ex}{\scalebox{.7}{$\frown$}}}%
}

\newcommand{\numberof}[1]{\ensuremath{\# #1}}
\newcommand{\samenumber}[2]{\ensuremath{\numberof{#1} \equiv \numberof{#2}}}

\renewcommand{\epsilon}{\varepsilon}

\newcommand{\iso}{\ensuremath{\cong}}
\newcommand{\gl}[2]{\ensuremath{\text{GL}\parens{#1, #2}}}
\newcommand{\abs}[1]{\ensuremath{\left| #1 \right|}}

%\newcommand{\reed}[1]{\relax}
%\newcommand{\Fix}[1]{\relax}
\newcommand{\reed}[1]{{\color{magenta}\bfseries [#1]}}
\newcommand{\Fix}[1]{{\color{red}\bfseries [#1]}}
\newcommand{\Comment}[1]{}
\newcommand{\Space}[1]{}
\newcommand{\Num}[1]{#1}

\newcommand{\injects}{\ensuremath{\hookrightarrow}}
\newcommand{\surjects}{\ensuremath{\twoheadrightarrow}}

\DeclarePairedDelimiter\ceil{\lceil}{\rceil}
\DeclarePairedDelimiter\floor{\lfloor}{\rfloor}

\newtheorem{theorem}{Theorem}
\numberwithin{theorem}{section}

\newtheorem{lemma}{Lemma}
\numberwithin{lemma}{section}

\newtheorem{proposition}{Proposition}
\numberwithin{proposition}{section}

\newtheorem{claim}{Claim}
\numberwithin{claim}{section}

\newtheorem{corollary}{Corollary}
\numberwithin{corollary}{section}

\newtheorem{definition}{Definition}
\numberwithin{definition}{section}

\newtheorem{example}{Example}
\numberwithin{example}{section}

\newcommand{\nat}{\ensuremath{\mathbb{N}}}
\newcommand{\real}{\ensuremath{\mathbb{R}}}
\newcommand{\integers}{\ensuremath{\mathbb{Z}}}
\newcommand{\rational}{\ensuremath{\mathbb{Q}}}
\newcommand{\complex}{\ensuremath{\mathbb{C}}}

\newcommand{\nonzero}[1]{\ensuremath{#1_{\neq 0}}}
\newcommand{\greaterset}[2]{\ensuremath{#1_{> #2}}}
\newcommand{\lesserset}[2]{\ensuremath{#1_{< #2}}}
\newcommand{\subgroup}[2]{\ensuremath{#1 < #2}}
\newcommand{\normalsubgroup}[2]{\ensuremath{#1 \triangleleft #2}}
\newcommand{\id}[1]{\ensuremath{\bm{1}_{#1}}}

\newcommand{\notationNonZero}[1][X]{Let $\nonzero{#1}$ denote the set of $x \in #1$ so that $x \neq 0$.\xspace}
\newcommand{\notationGreaterSet}[1][X]{Let $\greaterset{#1}{y}$ denote the set of $x \in #1$ so that $x > y$.\xspace}
\newcommand{\notationLesserSet}[1][X]{Let $\lesserset{#1}{y}$ denote the set of $x \in X$ so that $x < y$.\xspace}
\newcommand{\notationSubgroup}[1][H]{Let $\subgroup{#1}{G}$ denote that $#1$ is a subgroup of $G$.\xspace}
\newcommand{\notationNormalSubgroup}[1][N]{Let $\normalsubgroup{#1}{G}$ denote that $#1$ is a normal subgroup of $G$.\xspace}
\newcommand{\notationIdentity}[1][X]{Let $\id{#1}$ denote the identity function on $#1$.\xspace}

% \Sym{X} is the set of bijections from X to X
\newcommand{\Sym}[1]{\ensuremath{\text{Sym}\parens{#1}}}
\newcommand{\Aut}[1]{\ensuremath{\text{Aut}\parens{#1}}}
\newcommand{\notationSym}[1][X]{Let $\Sym{X}$ be the group of bijections of $X$.\xspace}

\newcommand{\toplim}{\ensuremath{\text{T}\lim}}
\newcommand{\toplimup}{\ensuremath{\overline{\toplim}}}
\newcommand{\toplimlow}{\ensuremath{\underline{\toplim}}}

\newcommand{\emptysingleton}{\ensuremath{\curlys{\emptyset}}}
\newcommand{\withoutempty}{\ensuremath{\setminus \emptysingleton}}

\newcommand{\isnat}[1]{\ensuremath{#1 \in \nat}}
\newcommand{\seq}[3]{\ensuremath{{\left( #1_{#2} \right)}_{#2 \in #3}}}
\newcommand{\setint}[3]{\ensuremath{\bigcap_{#2 \in #3} #1_{#2}}}
\newcommand{\natinti}[2]{\ensuremath{\setint{#1}{#2}{\nat}}}
\newcommand{\natint}[1]{\ensuremath{\natinti{#1}{n}}}
\newcommand{\natseqi}[2]{\ensuremath{\seq{#1}{#2}{\nat}}}
\newcommand{\natseq}[1]{\ensuremath{\natseqi{#1}{n}}}

\newcommand{\diam}[1]{\ensuremath{\text{diam} \left( #1 \right)}}

\newcommand{\eventually}{\ensuremath{\forall^{\infty}}}
\newcommand{\infinitelymany}{\ensuremath{\exists^{\infty}}}

\newcommand{\power}[1]{\ensuremath{\mathcal{P} \left( #1 \right)}}
\newcommand{\card}[1]{\ensuremath{\left| #1 \right|}}

\newcommand{\brackets}[3]{\ensuremath{{\left#1 {#3} \right#2}}}
\newcommand{\parens}[1]{\brackets{(}{)}{#1}}
\newcommand{\angles}[1]{\brackets{<}{>}{#1}}
\newcommand{\curlys}[1]{\brackets{\{}{\}}{#1}}
\newcommand{\squares}[1]{\brackets{[}{]}{#1}}
\newcommand{\parensum}[3]{\ensuremath{\parens{\sum_{#1}^{#2} {#3}}}}

\newcommand{\poly}[2]{\ensuremath{{#1}\left[ #2 \right]}}

\newcommand{\tand}{\ensuremath{~\text{and}~}}
\newcommand{\tor}{\ensuremath{~\text{or}~}}
\newcommand{\tsuchthat}{\ensuremath{~\text{s.t.}~}}

\newcommand{\beatree}[1]{\ensuremath{\subseteq {#1}^{\leq \nat}}}

\newcommand{\metricspace}[2]{\ensuremath{\parens{{#1},{#2}}}}

\newcommand{\setbuild}[2]{\ensuremath{\left\{ {#1} : {#2} \right\}}}

\newcommand{\openball}[3][]{\ensuremath{B_{#1}\parens{{#2},{#3}}}}
\newcommand{\closedball}[3][]{\ensuremath{\overline{B}_{#1}\parens{{#2},{#3}}}}

\newcommand{\powerset}[1]{\ensuremath{\mathcal{P}\parens{#1}}}
\newcommand{\powersetfin}[1]{\ensuremath{\mathcal{P}_{\text{fin}}\parens{#1}}}

\newcommand{\subgroupgen}[1]{\ensuremath{{\left< {#1} \right>}}}

\newcommand{\idmatrix}[1]{%
    \setcounter{n1}{#1 - 1}
    \begin{pmatrix}
        \forloop{i}{0}{\value{i} < #1}{%
            \forloop{j}{0}{\value{j} < #1}{%
                \ifthenelse{\equal{\value{i}}{\value{j}}}{1}{0}
                \ifthenelse{\value{j} < \value{n1}}{&}{}
            }
            \ifthenelse{\value{i} < \value{n1}}{\\}{}
        }
    \end{pmatrix}
}

\newcommand{\twobytwo}[4]{\ensuremath{\begin{pmatrix} #1 & #2 \\ #3 & #4 \end{pmatrix}}}

\newcommand{\after}{\circ}


\newcommand{\finitecommabody}[1]{}
\newcommand{\countablecommabody}[1]{}
\newcommand{\finitecomma}[2]{%
    \renewcommand{\finitecommabody}[1]{#1}%
    \ensuremath{\finitecommabody{0}, \finitecommabody{1}, \ldots, \finitecommabody{#2 - 1}}}
\newcommand{\finiteset}[2]{\ensuremath{\curlys{\finitecomma{#1}{#2}}}}
\newcommand{\countablecomma}[1]{\ensuremath{%
    \renewcommand{\countablecommabody}[1]{#1}%
    \ensuremath{\countablecommabody{0}, \countablecommabody{1}, \ldots}}}
\newcommand{\countableset}[1]{\ensuremath{\curlys{\countablecomma{#1}}}}

\newcommand{\woutlog}{without loss of generality\xspace}
\newcommand{\Woutlog}{Without loss of generality\xspace}

\newcommand{\actOn}[2]{\ensuremath{#1 \cdot #2}}
\newcommand{\kernel}[1]{\ensuremath{\text{ker}\parens{#1}}}

\newcommand{\deffuncparam}[6]{%
    \expandafter\newcommand\csname domain#1\endcsname{\ensuremath{#3}}%
    \expandafter\newcommand\csname codomain#1\endcsname{\ensuremath{#5}}%
    \expandafter\newcommand\csname decl#1\endcsname[1]{\ensuremath{#2_{##1} : #3 #4 #5}}%
    \expandafter\newcommand\csname funcname#1\endcsname[1]{\ensuremath{#2_{##1}}}%
    \expandafter\newcommand\csname body#1\endcsname[2]{\ensuremath{#6}}%
    \expandafter\newcommand\csname call#1\endcsname[2]{\ensuremath{#2_{##1} \parens{##2}}}%
    \expandafter\newcommand\csname callbody#1\endcsname[2]{\ensuremath{\expandafter\csname call#1\endcsname{##1}{##2} = \expandafter\csname body#1\endcsname{##1}{##2}}}%
    \expandafter\newcommand\csname map#1\endcsname[2]{\ensuremath{##2 \mapsto \expandafter\csname body#1\endcsname{##1}{##2} }}%
    \expandafter\newcommand\csname inversecall#1\endcsname[2]{\ensuremath{#2_{##1} \parens{##2}^{-1}}}%
    \expandafter\newcommand\csname image#1\endcsname[1]{\ensuremath{\text{im}\parens{#2_{##1}}}}%
}

\newcommand{\defbinfunc}[6]{%
    \expandafter\newcommand\csname domain#1\endcsname{\ensuremath{#3}}%
    \expandafter\newcommand\csname codomain#1\endcsname{\ensuremath{#5}}%
    \expandafter\newcommand\csname decl#1\endcsname{\ensuremath{#2 : #3 #4 #5}}%
    \expandafter\newcommand\csname funcname#1\endcsname{\ensuremath{#2}}%
    \expandafter\newcommand\csname body#1\endcsname[2]{#6}%
    \expandafter\newcommand\csname call#1\endcsname[2]{\ensuremath{#2 \parens{##1, ##2}}}%
    \expandafter\newcommand\csname callpair#1\endcsname[1]{\ensuremath{#2 \parens{##1}}}%
    \expandafter\newcommand\csname callbody#1\endcsname[2]{\ensuremath{\expandafter\csname call#1\endcsname{##1}{##2} = \expandafter\csname body#1\endcsname{##1}{##2}}}%
    \expandafter\newcommand\csname map#1\endcsname[2]{\ensuremath{\parens{##1, ##2} \mapsto \expandafter\csname body#1\endcsname{##1}{##2} }}%
}

\newcommand{\defbinop}[6]{%
    \expandafter\newcommand\csname domain#1\endcsname{\ensuremath{#3}}%
    \expandafter\newcommand\csname codomain#1\endcsname{\ensuremath{#5}}%
    \expandafter\newcommand\csname decl#1\endcsname{\ensuremath{#2 : #3 #4 #5}}%
    \expandafter\newcommand\csname funcname#1\endcsname{\ensuremath{#2}}%
    \expandafter\newcommand\csname body#1\endcsname[2]{#6}%
    \expandafter\newcommand\csname call#1\endcsname[2]{\ensuremath{##1 #2 ##2}}%
    \expandafter\newcommand\csname callbody#1\endcsname[2]{\ensuremath{\expandafter\csname call#1\endcsname{##1}{##2} = \expandafter\csname body#1\endcsname{##1}{##2}}}%
    \expandafter\newcommand\csname map#1\endcsname[2]{\ensuremath{\parens{##1, ##2} \mapsto \expandafter\csname body#1\endcsname{##1}{##2} }}%
}

\newcommand{\deffunc}[6]{%
    \expandafter\newcommand\csname domain#1\endcsname{\ensuremath{#3}}%
    \expandafter\newcommand\csname codomain#1\endcsname{\ensuremath{#5}}%
    \expandafter\newcommand\csname decl#1\endcsname{\ensuremath{#2 : #3 #4 #5}}%
    \expandafter\newcommand\csname funcname#1\endcsname{\ensuremath{#2}}%
    \expandafter\newcommand\csname body#1\endcsname[1]{#6}%
    \expandafter\newcommand\csname call#1\endcsname[1]{\ensuremath{#2 \parens{##1}}}%
    \expandafter\newcommand\csname callbody#1\endcsname[1]{\ensuremath{\expandafter\csname call#1\endcsname{##1} = \expandafter\csname body#1\endcsname{##1}}}%
    \expandafter\newcommand\csname map#1\endcsname[1]{\ensuremath{##1 \mapsto \expandafter\csname body#1\endcsname{##1} }}%
    \expandafter\newcommand\csname invcall#1\endcsname[1]{\ensuremath{#2^{-1}\parens{##1}}}%
    \expandafter\newcommand\csname image#1\endcsname{\ensuremath{\text{im}\parens{#2}}}%
}

\newcommand{\defproj}[3]{%
    \defbinfunc{#1#2}{\pi_{#2}}{#2 \times #3}{\to}{#2}{##1}%
    \defbinfunc{#1#3}{\pi_{#3}}{#2 \times #3}{\to}{#3}{##2}%
}


\input{../../../Reed/School/LaTeX/lemmas.tex}

\begin{document}

\begin{itemize}
    \item[] Reed Oei
    \item[] reedoei2
\end{itemize}

\begin{definition}
    Let $\alpha$ and $\gamma$ be ordinals.
    A sequence of sets $(A_{\alpha})_{\alpha < \gamma}$ is called \emph{monotone} if it is increasing (i.e., if $\alpha < \beta$, then $A_{\alpha} \subseteq A_{\beta}$ for all $\alpha, \beta < \gamma$) or decreasing; it is called \emph{strictly monontone} if it is strictly increasing or strictly decreasing.
\end{definition}

\begin{proposition}[Exercise 1]
    Let $X$ be a second-countable topological space and let $\alpha$ and $\gamma$ be ordinals.
    Then
    \begin{enumerate}[label=(\alph*)]
        \item $X$ has at most continuum-many open (resp. closed) sets.
        \item Any strictly monotone sequence $(U_{\alpha})_{\alpha < \gamma}$ of open (resp. closed) subsets of $X$ has countable length; that is, $\gamma$ is countable.
        \item Every montone sequence $(U_\alpha)_{\alpha < \omega_1}$ of open (resp. closed) subsets of $X$ eventually stabilizes; that is, there is some $\gamma < \omega_1$ such that for all ordinals $\alpha$ with $\gamma \leq \alpha < \omega_1$, we have $U_{\alpha} = U_{\gamma}$.
    \end{enumerate}
\end{proposition}
\begin{proof}
    Let $\mathcal{T}$ be the topology of $X$; because $X$ is second-countable, it has a countable basis $\mathcal{B}$.
    Then $\mathcal{T} \subseteq \setbuild{\bigcup A}{A \subseteq \mathcal{B}}$, so $|\mathcal{T}| \leq |2^{\mathcal{B}}| \leq |2^{\N}|$, as desired.
    Similarly, because each closed set is the complement of an open set (a bijection), there are at most continuum-many closed sets.

    Now let $(U_{\alpha})_{\alpha < \gamma}$ be a strictly monotone sequence of open subsets of $X$.
    Then every $U_{\alpha} = \bigcup A_{\alpha}$ for some $A_{\alpha} \subseteq \mathcal{B}$.
    Because $(U_{\alpha})_{\alpha < \gamma}$ is a strictly monotone, we must have either $U_{\alpha} \subsetneq U_{\beta}$ or $U_{\beta} \subsetneq U_{\alpha}$ for all $\alpha < \beta < \gamma$.
    We only consider the first case, because the second is very similar.
    Let $\alpha < \beta < \gamma$, so then we have $|A_{\alpha}| < |A_{\beta}|$.
    But because $\mathcal{B}$ is countable, we can only have countably many increases in the size of $A_{\alpha}$.
    But because these sets are in a bijection with the sets $U_{\alpha}$, there must only be countably many sets in the sequence $(U_{\alpha})_{\alpha < \gamma}$, as desired.
    Similarly, any strictly montone sequence of closed sets $(F_{\alpha})_{\alpha < \gamma}$ would give us a corresponding strictly monotone sequence of open sets $(F_{\alpha}^c)_{\alpha < \gamma}$, which must be of countable length by the previous argument.

    Finally, let $(V_{\alpha})_{\alpha < \omega_1}$ be a monotone sequence of open subsets of $X$.
    Again, we only consider the increasing case, because the decreasing case is very similar.
    Let $B_{\alpha} \subseteq \mathcal{B}$ such that $V_{\alpha} = \bigcup B_{\alpha}$; similarly, the sequence $(B_{\alpha})_{\alpha < \omega_1}$ must only increase in size countably many times.
    Then let $\gamma = \sup \setbuild{\alpha < \omega_1}{\exists \beta < \omega_1. B_{\alpha} \subsetneq B_{\beta}}$ be the supremum of such indexes where the sequence increases, which is countable by the regularity of $\omega_1$.
    Then for any $\alpha$ with $\gamma \leq \alpha < \omega_1$, we have
    \[
        U_{\alpha} = \bigcup B_{\alpha} = \bigcup B_{\gamma} = U_{\beta}
    \]
    Then considering a similar monotone sequence of closed subsets of $X$, we again construct a monotone sequence of open sets and apply the above argument.
\end{proof}

\begin{proposition}[Exercise 2 Remark]
    Any first-countable separable Hausdorff topological space  $X$ has cardinality at most continuum, but this fails when we drop the requirement that $X$ be first-countable.
\end{proposition}
\begin{proof}
    \reed{TODO}
\end{proof}

\begin{proposition}[Exercise 3]
    Let $X$ be a topological space.
    Consider the following statements
    \begin{enumerate}
        \item $X$ is a totally bounded metric space.
        \item $X$ is a separable metric space.
        \item $X$ is second countable.
        \item $X$ is Lindel\"of; i.e., every open cover has a countable subcover.
    \end{enumerate}
    Then have (1) $\implies$ (2) $\implies$ (3) $\implies$ (4).
\end{proposition}
\begin{proof}
    (1) $\implies$ (2): As $X$ totally bounded, let $F_{\epsilon}$ be a finite $\epsilon$-net of $X$.
    Then define $\mathcal{D} := \bigcup_{n \in \N} F_{1/n}$, which is a countable set.
    We wish to show it is dense, so let $U$ be a nonempty open subset of $X$.
    Let $n \in \N$ be such that $\frac{1}{n} < \diam{U}$.
    Then because $X \subseteq B(F_{1/n}, 1/n)$ there must be some $x \in F_{1/n}$ such that $B(x, 1/n) \cap U \neq \emptyset$.
    But then $x \in U$, so $\mathcal{D}$ is a countable dense subset, as required.

    (2) $\implies$ (3): Let $\mathcal{D} \subseteq X$ be a countable dense subset of $X$.
    Define $\mathcal{B} := \setbuild{B(x, 1/n)}{x \in \mathcal{D}, n \in \N}$.
    Now, suppose that $U \subseteq X$ is an open set.
    Then for every $x \in U$, there is some $n_x \in \N$ so that $B(x,1/n) \subseteq U$, and therefore $U = \bigcup_{x \in U} B(x, 1/n_x)$ and each $B(x, 1/n_x) \in \mathcal{B}$.
    So then $\mathcal{B}$ is a countable basis, as desired.

    (3) $\implies$ (4): Let $\mathcal{B}$ be a countable basis for $X$.
    Let $\mathcal{V}$ be an open cover of $X$, so $X = \bigcup \mathcal{V}$.
    For each $U \in \mathcal{B}$, pick some $V_U$ such that $U \subseteq V_U$, and define $\mathcal{U} := \setbuild{V_U}{U \in \mathcal{B}}$, which is a countable subcover of $\mathcal{V}$, as desired.
\end{proof}

We write $A \subseteq_c B$ and $B \supseteq_c A$ to mean that $\closure{A} \subseteq B$.

Define the \emph{Banach-Mazur} game $G^{**}(A)$ as follows for a subset $A$ of a Polish space $X$.
Let $\mathcal{W}$ be a countable weak basis for $X$, i.e., a countable collection of nonempty open sets such that every nonempty open set $U \subseteq X$ contains at least one $W \in \mathcal{W}$.
The a run of the Banach-Mazur game is a sequence $(U_0, V_0, U_1, V_1, \ldots)$ such that:
\begin{enumerate}[label=(\roman*)]
    \item $U_n, V_n \in \mathcal{W}$ for all $n$
    \item $\diam{U_n} < 1/n$ and $\diam{W_n} < 1/n$ for all $n$
    \item $U_0 \supseteq_c V_0 \supseteq_c U_1 \supseteq_c U_1 \supseteq_c V_1 \supseteq_c \ldots$
\end{enumerate}
Let $x \in X$ be such that $\curlys{x} = \bigcap_n \overline{U_n} = \bigcap_n \overline{V_n}$.
Then Player I wins if and only if $x \in A$.

\begin{proposition}\label{prop:forces-baire-meas}
    Let $X$ be a Baire space, and let $U \subseteq X$ be a nonempty open set.
    Let $A$ and $B$ be subsets of $X$.
    If $U \forces A$, and $A \forces B$, then $U \forces B$.
\end{proposition}
\begin{proof}
    We have both $U \setminus A$ and $A \setminus B$ are meager.
    Suppose that $U \setminus B$ is not meager.
    Then by the Baire alternative, there is some nonempty open set $V \subseteq X$ such that $V \forces (U \setminus B)$.
    So then $V \setminus (U \setminus B) = (V \setminus U) \cup (V \cap B)$ is meager, so in particular, $V \cap B$ is meager.
    But then $B$ is meager, so $A$ must also be meager (recall $A \setminus B$ is meager), and similarly $U$ must be meager.

    But $X$ is a Baire space, and $U$ is a nonempty open set, so $U$ is nonmeager, and this is a contradiction.
    So then $U \setminus B$ must be meager, and $U \forces B$, as desired.
\end{proof}

\begin{proposition}\label{prop:weak-basis-forces}
    Let $X$ be a Baire space, let $A_n \subseteq X$ be Baire measurable sets, and let $U \subseteq X$ be some nonempty open set.
    Then
    \[
        U \forces \bigcup_n A_n \iff \forall V \subseteq U \exists W \subseteq V \exists n (W \forces A_n)
    \]
    where $V$ and $W$ vary over some weak basis $\mathcal{W}$ for $X$.
\end{proposition}
\begin{proof}
    $(\implies):$ Suppose that $U \forces \bigcup_n A_n$.
    Let $V \subseteq U$ such that $V \in \mathcal{W}$.
    Because each $A_n$ is Baire measurable, there is some open set $U_n$ such that $A_n =^* U_n$.
    Because $V \subseteq U$, we have $V \forces \bigcup_n A_n$, and so then $V \forces \bigcup_n U_n$ by Proposition~\ref{prop:forces-baire-meas}.
    That is, $V \setminus (\bigcup_n U_n)$ is meager.
    But $V$ is a nonempty open set in a Baire space, so it is nonmeager.
    So then for some $U_n$, we must have $V \cap U_n \neq \emptyset$.
    Taking this $U_n$, because $V \cap U_n$ is nonempty and open, there is some $W \in \mathcal{W}$ such that $W \subseteq V \cap U_n$ and $W \forces A_n$, as required.

    ($\impliedby$): \reed{TODO, but actually I only need the one direction sufficient for it's use below}
\end{proof}

\begin{proposition}[Exercise 79]
    Let $X$ be a Polish space, $F \subseteq X \times \N^{\N}$ a closed set, and $A = \proj{X}(F)$.
        If Player II has a winning strategy in the Banach-Mazur game $G^{**}(F,X)$, then $A$ is meager.
\end{proposition}
\begin{proof}
    First, note that $A$ is analytic, and so it is universally measurable, and in particular, Baire measurable.

    Suppose that Player II has a winning strategy, but $A$ is not meager.
    Then because $X$ is a Polish space, it is also a Baire space, so there is some nonempty open set $U \subseteq X$ such that $U \forces A$.

    Let $U_0 \subseteq U$ be some nonempty open set in $\mathcal{W}$.
    Then $U_0 \forces A$ as well.
    Player II will pick some set $V_0 \subseteq_c U_0$.
    But because $U_0 \forces A$ and $V_0 \subseteq U_0$, we know that there is some set $U_1 \subseteq V_0$ such that $U_1 \forces A$ by Proposition~\ref{prop:weak-basis-forces}.
    Clearly we can pick $U_1 \subseteq V_0$ such that $\diam{U_1} < 1/n$ and $U_1 \subseteq_c V_0$, because $X$ is metrizable.
    Then simply repeat this process to build the whole run of the game.

    Each $U_i$ is nonempty and open, so they are all nonmeager because $X$ is a Baire space.
    Then because each $U_i \forces A$, that is, $U_i \setminus A$ is meager, we must have $U_i \cap A \neq \emptyset$ (in fact, it must be nonmeager), so then the $x \in X$ such that $\curlys{x} = \bigcap_n \overline{U_n}$ must also be $x \in A$.
    But then Player I wins, contradicting that Player II has a winning strategy.
    So it must be that $A$ is meager.
\end{proof}

\begin{lemma}
    Let $X$ and $Y$ be Polish spaces.
    If $f : X \to Y$ is universally measurable and $B \subseteq Y$ is Borel, then $f^{-1}(B)$ is universally measurable.
\end{lemma}
\begin{proof}
    This is obviously true for open sets, by the definition of a universally measurable function.
    If the $f$-preimage of some subset $A$ of $X$ is universally measurable, then the $f$-preimage of $A^c$ will also be universally measurable because universally measurable sets form a $\sigma$-algebra.
    Similarly, for a family of subsets $A_n$ of $X$ such that $f^{-1}(A_n)$ is universally measurable, the preimage $f^{-1}\parens{\bigcup_n A_n} = \bigcup_n f^{-1}(A_n)$ will also be universally measurable.
    So then the sets for which $f^{-1}(A)$ is universally measurable is a $\sigma$-algebra containing the open sets, and so contains every Borel set.
\end{proof}

\begin{lemma}
    Let $X$ and $Y$ be Polish spaces.
    If $f : X \to Y$ is universally measurable and $\mu$ is a $\sigma$-finite Borel measure on $X$, then the pushforward measure $f_* \mu$ is a $\sigma$-finite Borel measure on $Y$.
\end{lemma}
\begin{proof}
    Because $\mu$ is $\sigma$-finite, we have $X = \bigcup_n X_n$ where $\mu(X_n) < \infty$ for each $n$.
    To show that $f_*\mu$ is also $\sigma$-finite, let $Y_{n+1} = f(X_n)$, and let $Y_0 = Y \setminus \bigcup_n Y_n$, so $Y = \bigcup_n Y_n$, and $f_*\mu(Y_n) = \mu(X_n) < \infty$ for $n > 0$, and finally $f_*\mu(Y_0) = \mu(f^{-1}(Y_0)) = \mu(\emptyset) = 0 < \infty$.
\end{proof}

\begin{proposition}[Exercise 95]
    The composition of two universally measurable functions on Polish spaces is universally measurable.
\end{proposition}
\begin{proof}
    Let $X$, $Y$, and $Z$ be Polish spaces and let $f : X \to Y$ and $g : Y \to Z$ be universally measurable.
    Let $U \subseteq Z$ be an open set.

    By the universally measurability of $g$, we know that $g^{-1}(U)$ is universally measurable.
    Let $\mu$ be a $\sigma$-finite measure on $X$, and consider the pushforward measure $f_{*}\mu$ on $Y$, which is also $\sigma$-finite.
    Then $g^{-1}(U) =_{f_*\mu} B$ for some Borel $B \subseteq Y$.
    But $f^{-1}(B)$ is universally measurable, so then $f^{-1}(B) =_\mu C$ for some Borel $C \subseteq X$.
    Then
    \[
        \mu(C) = \mu(f^{-1}(B)) = f_*\mu(B) = f_*\mu(g^{-1}(U)) = \mu(f^{-1}(g^{-1}(U)))
    \]
    That is, $(g \after f)^{-1}(U) =_\mu C$, a Borel set, so $(g \after f)^{-1}(U)$ is universally measurable.
\end{proof}

\end{document}

