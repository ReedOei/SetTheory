\section{Counting}

Counting is one of the most basic of mathematical skills, and certainly one of the most useful.
Almost everything mathematical is in the pursuit of counting: counting the number of apples we have, the number of hours we need to work to buy something, the number of dollars we'll make from an investement, and the list goes on.
\reed{Justify more examples of counting}

What's the most basic form of counting?
Surely it must be counting things with whole numbers, so we'll start there.
Let's suppose I have some apples, but I don't know how many.
One way to figure out how many I have is, of course, to count.
It could be that I have one, two, three, four, five, six, or more apples.
There's also another case: I could have no apples at all.
In each case I have some number of apples: $0$, $1$, $2$, $3$, $4$, $5$, $6$, etc.

In fact, the case of no apples at all is somewhat curious.
Nowadays, we take for granted that $0$ is a number, just like any other---historically, this was not the case.
For a long time, people did math, and arithemetic, with no number zero at all.
And similarly, it may seem to you that we have no need for the number zero.
\reed{Justify the existence of zero}

We call all of these numbers, $0$, $1$, $2$, $3$, etc. ``natural numbers'' as they ``naturally'' arise in daily life; even disregarding all of technology and science, the first hunter-gatherers had to know how much food they had: how many berries, how many animals they hunted.
Early farming cultures needed to know how many days had passed to calculate when the rainy season would come.
All of these things are ``simply'' counting with natural numbers, but to say they are simple discounts the enormous leap in abstraction required to think like this.

\subsection{The Matter of Infinity}

One of the most central questions in mathematics is the study of infinite objects.
In fact, it is so central that in many cases, the so-called ``finite case'', where we only consider finite objects, is referred to as ``trivial'', meaning that it is so simple as to not be of interest.
We should briefly note that this is not \textbf{always} the case: there are many branches of mathematics that consider either exclusively or almost exclusively finite objects.
Infinite objects, despite their obvious lack of direct analogues in the so-called ``real world'' are incredibly pervasive.
The simplest of numbers, the natural numbers, are just such an infinite set.
The question of ``what is the last number?'' is easily answered with ``there is no last number'', and any set described in such a manner is almost \textbf{definitionally} infinite.

However, the question of infinity, and the myriad paradoxes that seem to plague the idea, pose many issues when we examine them.
In fact, what does infinity even mean?

Like all subjects, it is nearly impossible to discuss a subject intelligently before defining it.
People have vague intuitions of what infinity is and what it means; these different perspectives may be useful at different times.
However, each must be made precise so that we may determine when each persective is useful, and whether, in fact, it is a reasonable conception of infinity at all.
More generally, there is the question of what a good definition is; in some sense, we are seeking the definition of definitions.

Before we even define infinity, we will take a detour, delving into how we describe the size of anything at all, even finite sets.

