\section{Counting}

Counting is one of the most basic of mathematical skills, and certainly one of the most useful.
Almost everything mathematical is in the pursuit of counting: counting the number of apples we have, the number of hours we need to work to buy something, the number of dollars we'll make from an investement, and the list goes on.
\reed{Justify more examples of counting}

What's the most basic form of counting?
Surely it must be counting things with whole numbers, so we'll start there.
Let's suppose I have some apples, but I don't know how many.
One way to figure out how many I have is, of course, to count.
It could be that I have one, two, three, four, five, six, or more apples.
There's also another case: I could have no apples at all.
In each case I have some number of apples: $0$, $1$, $2$, $3$, $4$, $5$, $6$, etc.

In fact, the case of no apples at all is somewhat curious.
Nowadays, we take for granted that $0$ is a number, just like any other---historically, this was not the case.
For a long time, people did math, and arithemetic, with no number zero at all.
And similarly, it may seem to you that we have no need for the number zero.
\reed{Justify the existence of zero}

